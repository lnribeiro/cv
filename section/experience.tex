\section{Professional Experience}

{\bf Communications Research Laboratory, Technische Universit\"at Ilmenau}\\
Ilmenau, Germany  \hfill {\it November 2019 -- present} \\
\begin{itemize}
	\item[--] Postdoctoral fellow
	\item[--] Leads a research project on transceiver designs for 5G systems in collaboration with an industrial partner
	\item[--] Advises master students on research projects. The main research topic is the application of machine learning techniques to wireless channel prediction.
\end{itemize}

{\bf Wireless Telecommunications Research Group}\\
Fortaleza, Brazil \hfill {\it March 2016 -- October 2019}\\
\begin{itemize}
	\item[--] Research assistant
	\item[--] Developed low-complexity tensor-based beamforming filters for large-scale antenna systems. The proposed methods were implemented in MATLAB and the results were published in conferences and journal papers
	\item[--] Mentored undergraduate students
	\item[--] Teaching assistant: digital signal processing and communication systems
\end{itemize}

{\bf Tendência Edutech}\\
Fortaleza, Brazil \hfill {\it December 2018 -- November 2019}\\
\begin{itemize}
	\item[--] Data analysis consultant
	\item[--] Developed a dashboard to calculate and analyze the educational indicators of Recife (Brazil). The dashboard automatically downloads education-related microdata from the official Brazilian statistics database and calculates indicators which help to guide educational policies in Recife. The dashboard is implemented in R and Shiny
\end{itemize}

{\bf Christian Doppler Laboratory, Technische Universit\"at Wien}\\
Vienna, Austria \hfill {\it December 2016 -- December 2017} \\
\begin{itemize}
	\item[--] Erasmus Mundus visiting researcher 
	\item[--] Research on computationally efficient filter design and energy-efficient transceiver designs for modern mobile communication systems
\end{itemize}

{\bf I3S Laboratory, University of Nice-Sophia Antipolis}\\
Nice, France \hfill  {\it March 2014 -- July 2014} \\
\begin{itemize}
	\item[--] Research intern
	\item[--] Developed a tensor-based technique for extracting the atrial activity in atrial fibrillation electrocardiograms. Implemented the signal extraction technique in MATLAB and evaluated its performance in an electrocardiogram database obtained from patients with persistent atrial fibrillation.
\end{itemize}